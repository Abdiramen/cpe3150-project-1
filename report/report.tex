\RequirePackage[l2tabu, orthodox]{nag}
\documentclass[12pt]{article}

\usepackage{amssymb,amsmath,verbatim,graphicx,microtype,upquote,units,booktabs,siunitx,hyperref}
\setcounter{secnumdepth}{2}

\title{Project \#2}
\date{Due Date: November 8\textsuperscript{th}, 2016}
\author{Michael Schoen, Abdirahman Osman, Illya Starikov}

\newcommand{\br}{\\\multicolumn{2}{c}{} \\ }

\begin{document}
\maketitle

\noindent For our project, we have decided to implement a memory matching game. The game will start off by enabling a single light. If the user correctly presses said light, the user will move onto the next level and enable a light sequence. For every successive level, an additional light have to be kept in mind (i.e. for the $n$\textsuperscript{th} level, there will be $n$ lights that will have to be pressed).

This will continue for \num{15} games, where upon a successful finish, a special light sequence and a song will asynchronously play. From here, the user can choose to play another game if they so choose.

\section{Explanation}
Below we will take a more in-depth look into the implementation of our project.

\subsection{Random Number Generation}
Because we want our game to be different every round, we have to have some form of random number generation. Initially, we had tried to implement a \textit{linear congruential generator}; unfortunately, we were unsuccessful. We instead went with a different implementation.

Both implementations require a seed value (and to have different results, we need a different seed value) --- we ``cheat'' to get this. During our input, we not only check to see if the user initiated a game, but we also increment the register the seed is stored in. This way, there is only a $\nicefrac{1}{255}$ chance the user will get the same game\footnote{Although this is a significant value (\num{.39}\%), it serves well for demo purposes.}.

\subsubsection{Failed Attempt}
The basic summary is as follows: beginning with a seed value (named $X_n$), a new linear, psuedo-random number can be calculated via:

\begin{equation}
    X_{n + 1} = (a X_n + c) \mod m
\end{equation}

\noindent where the follow conditions hold

\begin{enumerate}
    \item $a, X_n, c, m \in \mathbb{Z}^+$
    \item $0 \leq X_n, c < m$
    \item $0 < a < m$
    \item $m \neq 0$
\end{enumerate}

The issue we face is it is \textit{highly} recommended that $m$ be quite large (most popular implementations range from $2^{31}$ to $2^{48}$); unfortunately, we only have $2^8$ available to us. Because of this, we either get a cycle roughly every \num{10} iterations or the same number produced. As one might imagine, this is a game-breaking bug; we decided to go with something different.

Our failed attempt for a linear congruential generator is as follows.

\begin{verbatim}
; Check writeup for how this works
; For now, the formula is

; X_n+1 = aX_n + c mod m

; For our purposes, a = 7, c = incrementor, m = 72
; for our purposes, incrementor can't be a multiple of 7
; X_n is stored in R7, incrementor in R6

; Result will be stored in A
RNG:
                MOV A, R7
                MOV B, #7D
                DIV AB
                MOV A, B

                JNZ SKIP1
                INC R6
SKIP1:          MOV A,  #7D
                MOV B, R7
                MUL AB                      ; A = aX_n

                ADD A, R6
                MOV B, #72D
                DIV AB

                MOV R7, B

                MOV A, B
                MOV B, #10D
                DIV AB

                MOV A, B
                RET
\end{verbatim}

\subsubsection{Successful Attempt}
For our successful attempt, we ported a random number generating from an open source code base\footnote{\url{https://www.pjrc.com/tech/8051/}}. It similar to the linear congruential generator in the sense that it linearly produces a psuedo-random number; however, it is a different formula (one that can't be eloquently described in an equation).

\subsection{Music}
We know the frequency of the Philips P89LPC932A1 to be \SI{7.373}{\mega\hertz}, with \SI{2}{cycles} per machine cycle. Therefore,

\begin{equation}
    \frac{2 \text{ cycles}}{\text{machine cycle}} \cdot \frac{1 \text{ Period}}{\SI{7.373}{\mega\hertz}} = \SI{0.27126}{\nicefrac{\micro\second}{mc}}
\end{equation}

\noindent We use this calculation as the base of our music.

\begin{center}
    \begin{tabular}{c|l}
                    & $f = \SI{659.255}{\hertz} \implies T$ = \SI{1516}{\micro\second} \\
    \textbf{E5}     & $\SI{1516}{\micro\second} \div \SI{.27126}{\nicefrac{\micro\second}{mc}} = \SI{5589}{mc}$  \\
                    & $\SI{5589}{mc} \div 4 = \SI{1398}{mc}$ \\
                    & $\SI{1398}{mc} \implies \num{699} \text{ iterations (with \texttt{DJNZ})}$ \br

                    & $f = \SI{698.456}{\hertz} \implies T$ = \SI{1431}{\micro\second} \\
    \textbf{F5}     & $\SI{1431}{\micro\second} \div \SI{.27126}{\nicefrac{\micro\second}{mc}} = \SI{5275}{mc}$  \\
                    & $\SI{5589}{mc} \div 4 = \SI{1318}{mc}$ \\
                    & $\SI{1318}{mc} \implies \num{569} \text{ iterations (with \texttt{DJNZ})}$ \br

                    & $f = \SI{783.991}{\hertz} \implies T$ = \SI{1275.5}{\micro\second} \\
    \textbf{G5}     & $\SI{1275.5}{\micro\second} \div \SI{.27126}{\nicefrac{\micro\second}{mc}} = \SI{4702}{mc}$  \\
                    & $\SI{4702}{mc} \div 4 = \SI{1176}{mc}$ \\
                    & $\SI{1176}{mc} \implies \num{588} \text{ iterations (with \texttt{DJNZ})}$ \br

                    & $f = \SI{587.330}{\hertz} \implies T$ = \SI{1702.6}{\micro\second} \\
    \textbf{D5}     & $\SI{1702.6}{\micro\second} \div \SI{.27126}{\nicefrac{\micro\second}{mc}} = \SI{6277}{mc}$  \\
                    & $\SI{7045}{mc} \div 4 = \SI{1570}{mc}$ \\
                    & $\SI{1570}{mc} \implies \num{785} \text{ iterations (with \texttt{DJNZ})}$ \br

                    & $f = \SI{523.251}{\hertz} \implies T$ = \SI{1911.1}{\micro\second} \\
    \textbf{C5}     & $\SI{1911.1}{\micro\second} \div \SI{.27126}{\nicefrac{\micro\second}{mc}} = \SI{7045}{mc}$  \\
                    & $\SI{7045}{mc} \div 4 = \SI{1760}{mc}$ \\
                    & $\SI{1760}{mc} \implies \num{880} \text{ iterations (with \texttt{DJNZ})}$ \br

                    & $f = \SI{554.365}{\hertz} \implies T$ = \SI{1803.8}{\micro\second} \\
    \textbf{Flat D5}& $\SI{1803.8}{\micro\second} \div \SI{.27126}{\nicefrac{\micro\second}{mc}} = \SI{6650}{mc}$  \\
                    & $\SI{6650}{mc} \div 4 = \SI{1662}{mc}$ \\
                    & $\SI{1662}{mc} \implies \num{831} \text{ iterations (with \texttt{DJNZ})}$ \br

    \end{tabular}
\end{center}

\section{Future Work}
\section{Work Effort}
\begin{itemize}
    \item Michael Schoen
    \begin{itemize}
        \item Programmed binary counter.
        \item Programmed game logic.
    \end{itemize}

    \item Osman Abdirahman
    \begin{itemize}
        \item Programmed initial beep.
        \item Programmed song implementation.
    \end{itemize}

    \item Illya Starikov
    \begin{itemize}
        \item Programmed initial beep.
        \item Programmed random number renerator.
        \item Programmed light sequence.
    \end{itemize}
\end{itemize}

\end{document}
